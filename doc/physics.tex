\documentclass[10pt,a4paper,twoside]{article}
\usepackage[left=2.45cm,top=2.52cm,right=1.85cm,bottom=2.95cm]{geometry}

\usepackage{hyperref}

\usepackage{amsmath}

\title{3D Game Physics}
\author{Jan Wedekind}
\date{November 25, 2017}

\begin{document}
\maketitle

\section{Introduction}
This technical report is about developing a small simulator with rigid bodies and soft body physics.

\section{Motion}
\subsection{One-dimensional motion}
\subsubsection{Constant position}
The position of each object in Euclidean space is a function of time $x(t)$.
The most basic case is a stationary object which is at a fixed position $x_0$ in the frame of reference.
\begin{equation*}
  x(t) = x_0
\end{equation*}
In this case the speed $v(t)$ and acceleration $a(t)$ are zero.
Speed and acceleration are the first and second derivative of the position over time (here denoted by one or two dots above the letter $x$).
\begin{align*}
  \begin{split}
    v(t) & = \dot{x}(t)  = 0\\
    a(t) & = \ddot{x}(t) = 0
  \end{split}
\end{align*}
Position is measured in meters ($m$), speed is measured in meters per second ($\frac{m}{s}$), and acceleration is measured in meters per second square ($\frac{m}{s^2}$).

\subsubsection{Linear motion}
A linear motion uses a constant speed and a position which changes linearly.
\begin{align*}
  \begin{split}
    x(t) & = x_0 + v_0 \cdot t\\
    v(t) & = v_0\\
    a(t) & = 0
  \end{split}
\end{align*}

\subsubsection{Constant acceleration}
If there is a constant acceleration, the speed will change linearly and the position is a quadratic function of time.
\begin{align*}
  \begin{split}
    x(t) & = x_0 + v_0 \cdot t + \frac{1}{2} \cdot a_0 \cdot t^2\\
    v(t) & = v_0 + a_0 \cdot t\\
    a(t) & = a_0
  \end{split}
\end{align*}

The acceleration is caused by applying a force $F$ to a mass $m$.
The acceleration is determined by dividing the force by the mass as follows:
\begin{equation*}
  a(t)=\frac{F(t)}{m}
\end{equation*}

\subsubsection{Mass spring-damper model}
The force of a spring is approximately proportional to the deviation from the resting position (Hooke's law).
Assuming the resting position is $0$, the force exerted by a spring with strength $k$ is:
\begin{equation*}
  F_k(t)=-k\dot x(t)
\end{equation*}
Damping is modelled by a force proportional to the speed:
\begin{equation*}
  F_c(t)=-c\dot v(t)
\end{equation*}
Using the sum of forces, the acceleration is
\begin{equation*}
  \ddot{x}(t)=-\frac{k}{m} x(t)-\frac{c}{m}\dot{x}(t)
\end{equation*}
There are closed solutions to this differential equation\footnote{see \url{https://en.wikipedia.org/wiki/Harmonic_oscillator} for more information} but for the purpose of implementing a simulator we are interested in a numerical solution where the conditions can be changed interactively.

% Euler integration
% Verlet integration
% Runge-Kutta integration

% 3D motion

% test on circular motion
% append Scheme code
% mass, force, inertia
% rigid body rotation
% 3d rotation

% gravitation

\end{document}
