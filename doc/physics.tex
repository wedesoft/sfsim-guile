\documentclass[12pt,a4paper,twoside]{article}
\usepackage[left=2.45cm,top=2.52cm,right=1.85cm,bottom=2.95cm]{geometry}

\usepackage{hyperref}

\usepackage{amsmath}

\usepackage{txfonts}

\title{3D Game Physics}
\author{Jan Wedekind}
\date{August 8, 2018}

\begin{document}
\maketitle

\section{Introduction}
This technical report is about developing a small simulator with rigid bodies and soft body physics.

\section{Motion}
\subsection{One-dimensional motion}
\subsubsection{Constant position}
The position of each object in Euclidean space is a function of time $x(t)$.
The most basic case is a stationary object which is at a fixed position $x_0$ in the frame of reference.
\begin{equation*}
  x(t) = x_0
\end{equation*}
In this case the speed $v(t)$ and acceleration $a(t)$ are zero.
Speed and acceleration are the first and second derivative of the position over time (here denoted by one or two dots above the letter $x$).
\begin{align*}
  \begin{split}
    v(t) & = \dot{x}(t)  = 0\\
    a(t) & = \ddot{x}(t) = 0
  \end{split}
\end{align*}
Position is measured in meters ($m$), speed is measured in meters per second ($\frac{m}{s}$), and acceleration is measured in meters per second square ($\frac{m}{s^2}$).

\subsubsection{Linear motion}
A linear motion uses a constant speed and a position which changes linearly.
\begin{align*}
  \begin{split}
    x(t) & = x_0 + v_0 \cdot t\\
    v(t) & = v_0\\
    a(t) & = 0
  \end{split}
\end{align*}

\subsubsection{Constant acceleration}
If there is a constant acceleration, the speed will change linearly and the position is a quadratic function of time.
\begin{align*}
  \begin{split}
    x(t) & = x_0 + v_0 \cdot t + \frac{1}{2} \cdot a_0 \cdot t^2\\
    v(t) & = v_0 + a_0 \cdot t\\
    a(t) & = a_0
  \end{split}
\end{align*}

\subsubsection{Force and momentum}
The acceleration is caused by applying a force $F$ to a mass $m$.
The acceleration is determined by dividing the force by the mass as follows:
\begin{equation*}
  a(t)=\frac{F(t)}{m}
\end{equation*}
\emph{I.e.} the heavier the object, the more force is required to accelerate it.

The momentum $p$ of an object is defined as the product of speed $v$ and mass $m$.
The speed can be determined by dividing the momentum by the mass as follows:
\begin{equation*}
  v(t)=\frac{p(t)}{m}
\end{equation*}

\subsubsection{Mass spring-damper model}
The force of a spring is approximately proportional to the deviation from the resting position (Hooke's law).
Assuming the resting position is $0$, the force exerted by a spring with strength $k$ is:
\begin{equation*}
  F_k(t)=-k \dot x(t)
\end{equation*}
Damping is modelled by a force proportional to the speed:
\begin{equation*}
  F_c(t)=-c \dot v(t)
\end{equation*}
Using the sum of forces, the acceleration is
\begin{equation*}
  \ddot{x}(t) = -\frac{k}{m} x(t) - \frac{c}{m}\dot{x}(t)
\end{equation*}
There are closed solutions to this differential equation\footnote{see \url{https://en.wikipedia.org/wiki/Harmonic_oscillator} for more information}, but for the purpose of implementing a simulator we are interested in a numerical solution where the conditions can be changed interactively.

\subsection{Numerical integration}
\subsubsection{Euler integration}
Euler integration is a popular method for numeric integration.
Applied to the motion model above, an Euler integration step for a time interval $\Delta t$ can be performed as follows:
\begin{align*}
  \begin{split}
    v(t + \Delta t) &= v(t) + a(t) \Delta t\\
    x(t + \Delta t) &= x(t) + v(t) \Delta t + \frac{1}{2} a(t) \Delta t^2
  \end{split}
\end{align*}

\subsubsection{Verlet integration}
Verlet integration is similar to Euler integration. Instead of the speed, the last two positions are used for the integration step:
\begin{equation*}
  x(t + \Delta t) = 2 x(t) - x(t - \Delta t) + a(t) \Delta t^2
\end{equation*}

\subsubsection{Runge-Kutta integration}
Both Euler and Verlet integration are not sufficiently accurate for integrating functions with large higher-order gradients such as encountered when simulating 3D rotations.
The Runge Kutta method is a popular numerical integration method for problems requiring higher accuracy\footnote{also see \url{https://en.wikipedia.org/wiki/Runge\%E2\%80\%93Kutta_methods}}.
Given an initial value $y(t_0)=y_0$ and a derivative $\dot{y}=f(t,y)$, the Runge-Kutta integration is performed as follows:
\begin{align*}
  \begin{split}
    k_1 &= f(t, y(t))\\
    k_2 &= f(t + \frac{\Delta t}{2}, y(t) + \frac{\Delta t}{2} k_1)\\
    k_3 &= f(t + \frac{\Delta t}{2}, y(t) + \frac{\Delta t}{2} k_2)\\
    k_4 &= f(t + \Delta t, y(t) + \Delta t k_3)\\
    y(t + \Delta t) &= y(t) + \frac{\Delta t}{6} (k_1 + 2 k_2 + 2 k_3 + k_4)
  \end{split}
\end{align*}

In order to perform numerical integration for both position and speed, one can set  $y(t) = (y_x(t), y_v(t))$ with $y_x(t)= x(t)$ and $y_v(t) = v(t)$.
\emph{I.e.} the state vector $y$ is composed of position and speed.
The initial state is set to
\begin{equation*}
  y(t_0) = (y_x(t_0), y_v(t_0))
\end{equation*}
The derivative of the state vector is
\begin{equation*}
  \dot{y}(t, y) = (y_v, a(t))
\end{equation*}

\section{Three-dimensional motion}
\subsection{Vectors}
In the three-dimensional case, position, speed, and motion are three-dimensional vectors.
\begin{equation*}
  \vec{x}(t) = \begin{pmatrix} x_1 \\ x_2 \\ x_3 \end{pmatrix}\mathrm{,\ }
  \vec{v}(t) = \begin{pmatrix} v_1 \\ v_2 \\ v_3 \end{pmatrix}\mathrm{,\ and\ }
  \vec{a}(t) = \begin{pmatrix} a_1 \\ a_2 \\ a_3 \end{pmatrix}
\end{equation*}

\subsubsection{Angular momentum and torque}
The angular momentum $\vec{L}$ is determined using the rotational inertial matrix $\mathcal{I}$ and the angular speed $\vec{\omega}$.
\begin{equation*}
  \vec{L}(t) = \mathcal{R}(t) \mathcal{I} \mathcal{R}^\top(t) \vec{\omega}(t)
\end{equation*}
$\mathcal{R}(t)$ is the rotation matrix at time $t$.
\emph{I.e.} the inertial matrix is defined in the object's coordinate system\footnote{also see \url{http://www.cs.unc.edu/~lin/COMP768-F07/LEC/rbd1.pdf}}.

The torque $\tau$ is the derivative of the angular momentum.
\begin{equation*}
  \vec{\tau}(t) = \dot{\vec{L}}(t)
\end{equation*}
The sum of forces causes a linear acceleration of the object
\begin{equation*}
  \vec{F}(t) = \sum \vec{F}_i(t)
\end{equation*}
while the torque is the remaining rotational contribution the forces generate.
\begin{equation*}
  \vec{\tau}(t) = \sum \vec{r}_i \times \vec{F}_i(t)
\end{equation*}
% test on circular motion
% append Scheme code
% mass, force, inertia
% rigid body rotation
% 3d rotation

% gravitation

\end{document}
